% Options for packages loaded elsewhere
\PassOptionsToPackage{unicode}{hyperref}
\PassOptionsToPackage{hyphens}{url}
%
\documentclass[
]{article}
\usepackage{amsmath,amssymb}
\usepackage{lmodern}
\usepackage{iftex}
\ifPDFTeX
  \usepackage[T1]{fontenc}
  \usepackage[utf8]{inputenc}
  \usepackage{textcomp} % provide euro and other symbols
\else % if luatex or xetex
  \usepackage{unicode-math}
  \defaultfontfeatures{Scale=MatchLowercase}
  \defaultfontfeatures[\rmfamily]{Ligatures=TeX,Scale=1}
\fi
% Use upquote if available, for straight quotes in verbatim environments
\IfFileExists{upquote.sty}{\usepackage{upquote}}{}
\IfFileExists{microtype.sty}{% use microtype if available
  \usepackage[]{microtype}
  \UseMicrotypeSet[protrusion]{basicmath} % disable protrusion for tt fonts
}{}
\makeatletter
\@ifundefined{KOMAClassName}{% if non-KOMA class
  \IfFileExists{parskip.sty}{%
    \usepackage{parskip}
  }{% else
    \setlength{\parindent}{0pt}
    \setlength{\parskip}{6pt plus 2pt minus 1pt}}
}{% if KOMA class
  \KOMAoptions{parskip=half}}
\makeatother
\usepackage{xcolor}
\IfFileExists{xurl.sty}{\usepackage{xurl}}{} % add URL line breaks if available
\IfFileExists{bookmark.sty}{\usepackage{bookmark}}{\usepackage{hyperref}}
\hypersetup{
  pdftitle={Exploring the Confusing Partnership of Japan's LDP and Komeito},
  pdfauthor={Underland, Jake},
  hidelinks,
  pdfcreator={LaTeX via pandoc}}
\urlstyle{same} % disable monospaced font for URLs
\usepackage[margin=1in]{geometry}
\usepackage{graphicx}
\makeatletter
\def\maxwidth{\ifdim\Gin@nat@width>\linewidth\linewidth\else\Gin@nat@width\fi}
\def\maxheight{\ifdim\Gin@nat@height>\textheight\textheight\else\Gin@nat@height\fi}
\makeatother
% Scale images if necessary, so that they will not overflow the page
% margins by default, and it is still possible to overwrite the defaults
% using explicit options in \includegraphics[width, height, ...]{}
\setkeys{Gin}{width=\maxwidth,height=\maxheight,keepaspectratio}
% Set default figure placement to htbp
\makeatletter
\def\fps@figure{htbp}
\makeatother
\setlength{\emergencystretch}{3em} % prevent overfull lines
\providecommand{\tightlist}{%
  \setlength{\itemsep}{0pt}\setlength{\parskip}{0pt}}
\setcounter{secnumdepth}{-\maxdimen} % remove section numbering
\newlength{\cslhangindent}
\setlength{\cslhangindent}{1.5em}
\newlength{\csllabelwidth}
\setlength{\csllabelwidth}{3em}
\newlength{\cslentryspacingunit} % times entry-spacing
\setlength{\cslentryspacingunit}{\parskip}
\newenvironment{CSLReferences}[2] % #1 hanging-ident, #2 entry spacing
 {% don't indent paragraphs
  \setlength{\parindent}{0pt}
  % turn on hanging indent if param 1 is 1
  \ifodd #1
  \let\oldpar\par
  \def\par{\hangindent=\cslhangindent\oldpar}
  \fi
  % set entry spacing
  \setlength{\parskip}{#2\cslentryspacingunit}
 }%
 {}
\usepackage{calc}
\newcommand{\CSLBlock}[1]{#1\hfill\break}
\newcommand{\CSLLeftMargin}[1]{\parbox[t]{\csllabelwidth}{#1}}
\newcommand{\CSLRightInline}[1]{\parbox[t]{\linewidth - \csllabelwidth}{#1}\break}
\newcommand{\CSLIndent}[1]{\hspace{\cslhangindent}#1}
\ifLuaTeX
  \usepackage{selnolig}  % disable illegal ligatures
\fi

\title{Exploring the Confusing Partnership of Japan's LDP and Komeito}
\author{Underland, Jake}
\date{2022-07-15}

\begin{document}
\maketitle

{
\setcounter{tocdepth}{2}
\tableofcontents
}
\hypertarget{introduction}{%
\section{Introduction}\label{introduction}}

In conventional political theory, winning parties form coalitions when
they do not hold a single party majority in congress, but do so in such
a way as to minimize the cost of the coalition. This means partnering up
with a small partner that is needed to secure a majority in congress but
no more. This leaves the winning party at liberty to design and pass
legislation in their interests without conceding too much to the much
smaller, weaker coalition partner. However, this model does not provide
a satisfying explain to the puzzling nature of the long term LDP-Komeito
coalition. The LDP has been able to maintain a single party majority in
both Upper and Lower houses since 2016. Despite this, the LDP has
continued their partnership with Komeito since 2003.\\
\hspace*{0.333em}\hspace*{0.333em}\hspace*{0.333em}\hspace*{0.333em}In
this essay, I compare this peculiar case of Japan with Germany to
identify the reasons in which this unique phenomenon may occur. In this
comparison, Germany represents the control, being a parliamentary system
close in design to Japan with similar electoral rules and no such long
term alliance (besides the CDU/CSU alliance which is widely considered a
single party in the academic literature and will be assumed to be such
in this paper). Therefore, any difference between the German case and
the Japanese case is considered as highlighting the idiosyncrasy of the
LDP-Komeito alliance. I propose a new model of coalition formation which
begins not at the end of elections but before the start of the election
cycle. This model incorporates the costs and benefits of campaign
collaboration and electoral strategy to explain the rationale behind
this longstanding partnership.

\hypertarget{the-puzzle}{%
\section{The Puzzle}\label{the-puzzle}}

Traditional political theory explains coalition formation as an effort
made by parties which do not singly maintain a majority in parliament to
gain enough seats to pass legislation within their interests. This
theory, the Baron-Ferejohn model (Gehlbach 2013), treats the end of each
election cycle as the beginning of a bargaining game by parties. The
party with the largest share of seats (winning party) searches for a
party to form a coalition with and gain a majority in congress. Gaining
a majority of seats is not the only criteria for coalition formulation,
though, as partnerships with parties holding a large share of seats
would require greater concessions by the winning party with regards to
the content of legislation. In order for the winning party to earn the
gratifying position of policy maker while minimizing the magnitude of
concessions needed to appease their partnering party, the winning party
would pair up with a party that is large enough to gain a majority of
seats in congress but no larger than that. Extensions of this model
include considerations of policy preferences and ideological positions
by incorporate them into the cost of coalition formation (policy
preferences directly influence the magnitude of concessions, while
partnerships with ideologically opposed parties may alienate the winning
party's support base). In all of these extensions, there is no reason
for any party capable of maintaining a single-party majority in congress
to form a coalition with another party. Confusingly, this
counterintuitive case is what we observe in the LDP-Komeito coalition.

\includegraphics{final_files/figure-latex/unnamed-chunk-3-1.pdf}

~~~~Above is a bar chart showing the share of seats gained in Japan's
Lower House elections since 1990. The LDP-Komeito coalition began in
1999 and persisted until this day with the exception of the period of
time from 2009 when the Democratic Party was in power. We can see from
the graph that in its inception, the partnership of LDP and Komeito was
a necessary move for the LDP to hold a majority in congress.
Furthermore, the small size of Komeito is congruent with the
Baron-Ferejohn prediction regarding coalitions, and it is not hard to
imagine that the costs of the LDP pairing with either the JCP or SDP
(incurred from the parties' striking ideological and policy differences)
outweigh the extra 2\% of leverage granted to Komeito.\\
\hspace*{0.333em}\hspace*{0.333em}\hspace*{0.333em}\hspace*{0.333em}It's
from 2005 that things deviate from what the Baron-Ferejohn framework
predicts. In all Lower House elections after 2005 in which the LDP won,
they won with a single party majority. This is not limited to the Lower
House, as the LDP had also obtained a single party majority in the Upper
House in 2016 and most recently 2022. Regardless, the LDP has continued
its coalition with Komeito. When considering that the Komeito is largely
pacifist and reserved on the issue of constitutional revision, a key
policy interest of the LDP, this partnership is all the more bemusing
({``Minna to Watashi No Kenpou''} 2022). Indeed, prior to their
alignment, the two parties would frequently go head to head, even
leading to organized negative campaigning by the LDP regarding the
Komeito's religious background. In comparison, Germany's coalition
formation can be explained as being for the most part in accordance with
Baron-Ferejohn's theory of coalitions. \newpage

\begin{figure}[!h]
\includegraphics[width=\linewidth]{bundestag.png}%
\caption{Source: \url{https://commons.wikimedia.org/wiki/File:German_parliamentary_elections_diagram_de.svg}\textsuperscript{*}}
\small\textsuperscript{*} See here for the reliability of well maintained Wikipedia articles: \url{https://www.economist.com/international/2021/01/09/wikipedia-is-20-and-its-reputation-has-never-been-higher}
\end{figure}

~~~~The above figure shows the share of seats each party obtained in
Germany's \emph{bundestag} while the labels in the center indicate the
subsequently formed coalitions. There is a trend for the winning party
to partner with the FDP or Green Party, two small parties with weaker
influence, to secure majority in congress. The fact that the CDU/CSU and
SPD are both moderate parties means their coalitions with the centrist
FDP are ideologically viable as well. The only noticeable exceptions are
the \emph{grand coalitions} in which the CDU/CSU and SPD join forces to
form a coalition. This first occurred in 2005, when the SPD refused to
form a broader coalition with the Green Party and Party of Democratic
Socialism out of a strong aversion towards the latter party, due to
fears that their controversial ties to communist forces would alienate
support of the people. The 2013 grand coalition was formed on similar
grounds. In 2017, CDU/CSU broke off a possible coalition with FDP in
favor of a grand coalition due to disagreements on tax policy,
interpretable as a punitive measure against the FDP's growing demands in
their partnership. In 2017, again the CDU/CSU formed a grand coalition,
but not without considerable negotiation efforts seeking for an
alternative (Escritt 2018).\\
\hspace*{0.333em}\hspace*{0.333em}\hspace*{0.333em}\hspace*{0.333em}Even
the grand coalition cases of Germany highlight that negotiations occur
post election between parties, where they strategically form coalitions
to further their own interests. But for a party singularly holding a
majority stronghold in congress, what incentive is there to team up with
a smaller party and conceding policy positions, legislative content, and
important roles in government?

\hypertarget{extending-the-model-cost-of-collaboration}{%
\section{Extending the Model: Cost of
Collaboration}\label{extending-the-model-cost-of-collaboration}}

In order to explain this relationship, I extend the Baron-Ferejohn
coalition formation model to account for the costs and benefits of
campaign collaboration. In Japan and Germany's electoral systems, there
exists a mixture of single-member-district and proportional
representation voting. However, while in Japan the
single-member-district comprises the majority of seats, in Germany the
allocation of seats in the \emph{bundestag} are decided primarily by the
proportional representation portion. The significant implication of this
is that in Germany, strategic and cooperative campaigning/nominations of
candidates in single-member-districts is not as beneficial as it is in
Japan. Especially in recent years, Japanese opposition parties have
begun to strategize in elections by forming a unified opposition front
and nominating only one candidate from the front in the
single-member-districts so as not to split the vote. This indicates an
open acknowledgement of the influence of vote splitting in
single-member-districts and the subsequent wasted votes on election
prospects. ~~~~With this in mind, I propose an extended model of
coalition formation; that which begins not at the end of the election
cycle but at the beginning. Because in the Japanese electoral system
multiparty cooperation in election campaigning and candidate nominations
have considerable influence on the outcomes of the elections, it makes
sense to think that parties make decisions regarding the formation of
coalitions \emph{before} the election cycle starts. In our specific
case, the LDP and Komeito must have a long term agreement which dictates
the way in which they campaign.\\
\hspace*{0.333em}\hspace*{0.333em}\hspace*{0.333em}\hspace*{0.333em}Evidence
supports this. Since 2003, the LDP have officially backed all candidates
put forth by Komeito, and Komeito has officially backed most candidates
put forth by the LDP. Further, the two parties have abstained from
nominating candidates in the same electoral districts, leaving no room
for split votes affecting their chances. This strategy, while effective,
poses a problem: agreeing to withdraw candidates from specific districts
may cost potential seats for both parties. We can treat this as the cost
of electoral collaboration.\\
\hspace*{0.333em}\hspace*{0.333em}\hspace*{0.333em}\hspace*{0.333em}This
cost is the deciding factor in explaining the relationship of the LDP
and Komeito. One study from 2003, which analyzed party votes by
prefecture in the 2000 Lower House elections (previous to LDP and
Komeito's strategic cooperation in elections), shows that there was an
inverse correlation of -0.38 in the LDP and Komeito's performance in
prefectures, smaller than other parties of similar scale excluding the
JCP (Nakanishi 2003). What this implies is that the LDP and Komeito have
vastly different support bases, and in districts with high LDP support
Komeito support is low and vice-versa. It follows that the cost of
cooperation is significantly low for the LDP and Komeito. With few
districts in which the two parties both run as competitive candidates,
withdrawing candidates in the districts where their partner is likely to
win is a negligible cost. Conversely, if the LDP were to pair up with a
party that is supported in the same districts as them, electoral
cooperation will come with a steep price for at least one of the
parties. ~~~~Once considering the cost of collaboration, the LDP-Komeito
alliance can be explained as the rational decision of two strategic
actors. The lack of such long term alliances in Germany is due to the
differences in the electoral systems which make similar electoral
collaborations a fruitless exercise in Germany.

\hypertarget{conclusion}{%
\section{Conclusion}\label{conclusion}}

~~~~To summarize, the difference in support base of LDP and Komeito
presents a clear strategic advantage in elections which warrants their
collaboration. The framework I used to analyze this relationship has
interesting implications. First, it shows that the LDP-Komeito
partnership is not some unexplained anomaly but a rational decision that
can be explained by extending existing theories on coalition
formulation. Further, it highlights the unlikely mechanisms in electoral
systems that lead to moderated coalitions and may have contributed to a
well balanced cabinet in Japan, furthering checks and balances in a
country with a historically dominant party. \hfill (Word Count: 1771)

\hypertarget{references}{%
\section*{References}\label{references}}
\addcontentsline{toc}{section}{References}

\hypertarget{refs}{}
\begin{CSLReferences}{1}{0}
\leavevmode\vadjust pre{\hypertarget{ref-grand-coalition-2017}{}}%
Escritt, Thomas. 2018. {``Few Cheers at Home for Germany's Last-Resort
Coalition.''} \emph{Reuters}.

\leavevmode\vadjust pre{\hypertarget{ref-formal-methods}{}}%
Gehlbach, Scott. 2013. \emph{Formal Models of Domestic Politics}.
Analytical Methods for Social Research. Cambridge University Press.
\url{https://doi.org/10.1017/CBO9781139045544}.

\leavevmode\vadjust pre{\hypertarget{ref-komeito-kaiken}{}}%
{``Minna to Watashi No Kenpou.''} 2022. \emph{NHK}.
\url{https://www3.nhk.or.jp/news/special/minnanokenpou/seitou/koumei.html}.

\leavevmode\vadjust pre{\hypertarget{ref-ldp-komeito-support}{}}%
Nakanishi, Hiroko. 2003. {``What Percentage of Votes Won Per Prefecture
Can Tell Us.''} \emph{Nihon Operations Research Gakkai}.
\url{http://www.orsj.or.jp/~archive/pdf/bul/Vol.48_01_017.pdf}.

\end{CSLReferences}

\end{document}
